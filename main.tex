% !TEX root = ./notes_template.tex
%%%%%%%%%%%%%%%%%%%%%%%%%%%%%%%%%%%%%%%%%%%%%%%%%%
%%%%%%%%%%%%%%%%%%%%% preamble %%%%%%%%%%%%%%%%%%%
%%%%%%%%%%%%%%%%%%%%%%%%%%%%%%%%%%%%%%%%%%%%%%%%%%
\documentclass[11pt,twoside]{book}


\usepackage{luatex85}


\usepackage{ctex}
\renewcommand{\contentsname}{目录}
\usepackage{fontspec}
\usepackage{xeCJK}
\setCJKmainfont{LXGW WenKai Mono}
\linespread{1.5}


\usepackage{xeCJKfntef}
\xeCJKsetup{underdot/symbol={\normalfont^^b7}}
\newcommand{\dotemph}[1]{\CJKunderdot{#1}}





%\renewcommand{\baselinestretch}{1.05}
\usepackage{amsmath,amsthm,amssymb,mathrsfs,amsfonts,dsfont}
\usepackage{epsfig,graphicx}
\usepackage{tabularx}
\usepackage{blkarray}
\usepackage{slashed}
\usepackage{color}
\usepackage{listings}
\usepackage{caption}
% \usepackage{fullpage}
\usepackage{lipsum} % provides dummy text for testing
\usepackage[toc,title,titletoc,header]{appendix}
\usepackage{minitoc}
\usepackage{color}
\usepackage{multicol} % two-col ToC
\usepackage{bm}
\usepackage{imakeidx} % before hyperref
\usepackage{hyperref}
% link colors settings
\hypersetup{
    colorlinks=true,
    citecolor=magenta,
    linkcolor=magenta,
    filecolor=green,      
    urlcolor=cyan,
    % hypertexnames=false,
}
\usepackage[capitalise]{cleveref}
\usepackage{subcaption}
\usepackage{enumitem}
\usepackage{mathtools}
\usepackage{physics}
\usepackage[linesnumbered,ruled,vlined,algosection]{algorithm2e}
\SetCommentSty{textsf}
\usepackage{epigraph}
\epigraphwidth=1.0\linewidth
\epigraphrule=0pt

% adjust margin
\usepackage[margin=2.3cm]{geometry}
\headheight13.6pt

%%%%%%%%%%%%%%%% thmtools %%%%%%%%%%%%%%%%%%%%%
\usepackage{thmtools}
\declaretheorem[numberwithin=chapter, name=定理]{theorem}
\declaretheorem[numberwithin=chapter]{axiom}
\declaretheorem[numberwithin=chapter]{lemma}
\declaretheorem[numberwithin=chapter]{proposition}
\declaretheorem[numberwithin=chapter]{claim}
\declaretheorem[numberwithin=chapter]{conjecture}
\declaretheorem[sibling=theorem]{corollary}
\declaretheorem[numberwithin=chapter, style=definition, name=定义]{definition}
\declaretheorem[numberwithin=chapter, style=definition]{problem}
\declaretheorem[numberwithin=chapter, style=definition]{example}
\declaretheorem[numberwithin=chapter, style=definition]{exercise}
\declaretheorem[numberwithin=chapter, style=definition]{observation}
\declaretheorem[numberwithin=chapter, style=definition]{fact}
\declaretheorem[numberwithin=chapter, style=definition]{construction}
\declaretheorem[numberwithin=chapter, style=definition]{remark}
\declaretheorem[numberwithin=chapter, style=remark]{question}
%%%%%%%%%%%%%%%% thmtools %%%%%%%%%%%%%%%%%%%%%
\usepackage{changepage}
\newenvironment{solution}
    {\renewcommand\qedsymbol{$\square$}\color{blue}\begin{adjustwidth}{0em}{2em}\begin{proof}[\textit Solution.~]}
    {\end{proof}\end{adjustwidth}}

%%%%%%%%%%%%%%%% index %%%%%%%%%%%%%%%%%%%%%
\begin{filecontents}{index.ist}
% https://tex.stackexchange.com/questions/65247/index-with-an-initial-letter-of-the-group
headings_flag 1
heading_prefix "{\\centering\\large \\textbf{"
heading_suffix "}}\\nopagebreak\n"
delim_0 "\\nobreak\\dotfill"
\end{filecontents}
\newcommand{\myindex}[1]{\index{#1} \emph{#1}}
\makeindex[columns=3, intoc, title=Alphabetical Index, options= -s index.ist]
%%%%%%%%%%%%%%%% index %%%%%%%%%%%%%%%%%%%%%

%%%%%%%%%%%%%%%% ToC %%%%%%%%%%%%%%%%%%%%%
% Link Chapter title to ToC: https://tex.stackexchange.com/questions/32495/linking-the-section-text-to-the-toc
\usepackage[explicit]{titlesec}
% \titleformat{\chapter}[display]
%   {\normalfont\huge\bfseries}{\chaptertitlename\ {\thechapter}}{20pt}{\hyperlink{chap-\thechapter}{\Huge#1}
% \addtocontents{toc}{\protect\hypertarget{chap-\thechapter}{}}}
% \titleformat{name=\chapter,numberless}
%   {\normalfont\huge\bfseries}{}{-20pt}{\Huge#1}

  \titleformat{\chapter}[display]
  {\normalfont\huge\bfseries}{第\,\thechapter\,章}{20pt}{\hyperlink{chap-\thechapter}{\Huge#1}
\addtocontents{toc}{\protect\hypertarget{chap-\thechapter}{}}}


\titleformat{\part}[display]
{\centering\normalfont\huge\bfseries}{第\,\chinese{part}\,部分}{20pt}{\hyperlink{chap-\thepart}{\Huge#1}
\addtocontents{toc}{\protect\hypertarget{chap-\thepart}{}}}







\titleformat{\subsubsection}[runin]
  {\normalfont\large\bfseries}{}{}{#1}[]




%%%%%%%%%%%%%%%%%%% fancyhdr %%%%%%%%%%%%%%%%%
\usepackage{fancyhdr}
\pagestyle{fancy} % enable fancy page style
\renewcommand{\headrulewidth}{0.0pt} % comment if you want the rule
\fancyhf{} % clear header and footer
\fancyhead[lo,le]{\leftmark}
\fancyhead[re,ro]{\rightmark}
\fancyfoot[CE,CO]{\hyperref[toc-contents]{\thepage}}

% https://tex.stackexchange.com/questions/550520/making-each-page-number-link-back-to-beginning-of-chapter-or-section
\makeatletter
\def\chaptermark#1{\markboth{\protect\hyper@linkstart{link}{\@currentHref}{Chapter \thechapter ~ #1}\protect\hyper@linkend}{}}
\def\sectionmark#1{\markright{\protect\hyper@linkstart{link}{\@currentHref}{\thesection ~ #1}\protect\hyper@linkend}}
\makeatother
%%%%%%%%%%%%%%%%%%% fancyhdr %%%%%%%%%%%%%%%%%


%%%%%%%%%%%%%%%%%%% biblatex %%%%%%%%%%%%%%%%%
\usepackage[doi=false,url=false,isbn=false,style=alphabetic,backend=biber,backref=true, minalphanames=3]{biblatex}

\DefineBibliographyStrings{english}{
  backrefpage={Cited on page},
  backrefpages={Cited on pages}
}

\addbibresource{bib.bib}

\DefineBibliographyStrings{english}{bibliography={参考文献}}

\newbibmacro{string+doiurlisbn}[1]{%
  \iffieldundef{doi}{%
    \iffieldundef{url}{%
      \iffieldundef{isbn}{%
        \iffieldundef{issn}{%
          #1%
        }{%
          \href{http://books.google.com/books?vid=ISSN\thefield{issn}}{#1}%
        }%
      }{%
        \href{http://books.google.com/books?vid=ISBN\thefield{isbn}}{#1}%
      }%
    }{%
      \href{\thefield{url}}{#1}%
    }%
  }{%
    \href{http://dx.doi.org/\thefield{doi}}{#1}%
  }%
}



% https://tex.stackexchange.com/questions/94089/remove-quotes-from-inbook-reference-title-with-biblatex
\DeclareFieldFormat[article,incollection,inproceedings,book,misc]{title}{\usebibmacro{string+doiurlisbn}{\mkbibemph{#1}}}
% https://tex.stackexchange.com/questions/454672/biblatex-journal-name-non-italic
\DeclareFieldFormat{journaltitle}{#1\isdot}
\DeclareFieldFormat{booktitle}{#1\isdot}
% https://tex.stackexchange.com/questions/10682/suppress-in-biblatex
\renewbibmacro{in:}{}
% add video field: https://tex.stackexchange.com/questions/111846/biblatex-2-custom-fields-only-one-is-working
\DeclareSourcemap{
    \maps[datatype=bibtex]{
      \map{
        \step[fieldsource=video]
        \step[fieldset=usera,origfieldval]
    }
  }
}
\DeclareFieldFormat{usera}{\href{#1}{\textsc{Online video}}}
\AtEveryBibitem{
    \csappto{blx@bbx@\thefield{entrytype}}{% put at end of entry
        \iffieldundef{usera}{}{\space \printfield{usera}}
    }
}
%%%%%%%%%%%%%%%%%%% biblatex %%%%%%%%%%%%%%%%%

%%%%%%%%%%%%%%%%%%%%% glossaries %%%%%%%%%%%%%%%%%
% !TEX root = ./notes_template.tex
% \usepackage[style=super]{glossaries}
% https://www.overleaf.com/learn/latex/Glossaries
\usepackage[style=super,toc,acronym]{glossaries}
\setlength{\glsdescwidth}{1\linewidth}
\makeglossaries

\renewcommand\glossaryname{List of Abbreviations and Symbols}

\newglossaryentry{Q2}{name={$Q_2(f)$},
%sort=Q2,
description={Two-side (bounded) error quantum query complexity}}

\newglossaryentry{real_number}{name={$\mathbb{R}$},description={Real number}}

% \newglossaryentry{gcd}{name={gcd},description={greatest common divisor}}

\newacronym{gcd}{GCD}{Greatest Common Divisor}


\newglossaryentry{svm}{name={SVM},description={Support Vector Machine}}

\newglossaryentry{gd}{name={GD},description={Gradient Descent}}

\newglossaryentry{qft}{name={QFT},description={Quantum Field Theory}}

\newglossaryentry{qm}{name={QM},description={Quantum Mechanics}}

\newglossaryentry{v}{name={$\vec{v}$},description={a vector}}

% physics
\newglossaryentry{hamiltonian}{name={$\hat{H}$},description={Hamiltonian}}

\newglossaryentry{lagrangian}{name={$L$},description={Lagrangian}}
%%%%%%%%%%%%%%%%%%%%% glossaries %%%%%%%%%%%%%%%%%

%%%%%%%%%%%%%%%%%%%%% glossaries-extra %%%%%%%%%%%%%%%%%
% \usepackage[record,abbreviations,symbols,stylemods={list,tree,mcols}]{glossaries-extra}
%%%%%%%%%%%%%%%%%%%%% glossaries-extra %%%%%%%%%%%%%%%%%


% !TEX root = ./notes_template.tex

%%%%%%%%%%%%%%%%%%%%%%%%%%%%%%%%%%%%
%%%%%%%%%%%%%%%%%%%%%%%%%%%%%%%%%%%%
% math
\let\iff\relax
\newcommand{\iff}{\text{ iff }}
\newcommand{\OPT}{\textup{OPT}}

% physics
\newcommand{\acreation}{a^\dagger}





%%%%%%%%%%%%%%%%%%%%%%%%%%%%%%%%%%%%%%%%%%%%%%%%%%
%%%%%%%%%%%%%%%% begin of document %%%%%%%%%%%%%%%
%%%%%%%%%%%%%%%%%%%%%%%%%%%%%%%%%%%%%%%%%%%%%%%%%%

\begin{document}

\title{\bf \huge 基于秘密共享的安全多方计算简介}
\author{Kurt Pan}
\date{\today}
\maketitle
\setcounter{tocdepth}{2}
\setcounter{minitocdepth}{1} 

% \begin{multicols}{2}
    \dominitoc% Initialization
    \adjustmtc[2]% chp number shift for mini-toc
    \tableofcontents
    \label{toc-contents}
% \end{multicols}

% 	\listoffigures
% 	% \listoftables
% \begin{multicols}{2}
% 	\listoftheorems[ignoreall,show={theorem}]
% \end{multicols}

% 	\renewcommand{\listtheoremname}{List of Definitions}
% \begin{multicols}{2}
% 	\listoftheorems[ignoreall,show={definition}]
% \end{multicols}

	% \printglossaries
	% \printglossary[type=\acronymtype]
	% \printglossary
	% \printglossary[title=List of terms, toctitle=List of terms]

	% bib2gls
	% \printunsrtglossaries % print all types
	% \printunsrtglossary[type={abbreviations},title=List of Abbreviations,style=listgroup]
	% \printunsrtglossary[type={abbreviations},title=List of Abbreviations,style=listhypergroup] % doesn't work
	% \printunsrtglossary[type={symbols},title=List of Symbols,style=listgroup]
	% \printunsrtglossary % main entry

%%%%%%%%%%%%%%%Content%%%%%%%%%%%%%%%

\mainmatter % separat the number of toc and mainmatter


\chapter*{引言}\label{chp:00intro}
\addcontentsline{toc}{chapter}{引言}


A big part of the theory and practice of cryptography is devoted to the study of different technologies deployed in our world today. Standard topics include the task of encryption, which relates to hiding information, but it is also common to consider digital signatures and message authentication codes to ensure integrity, enable authentication and authorization, and many other subjects relevant for today’s infrastructure. For centuries, these were essentially the main tasks associated with the idea of securing information and communi- cation, and this continues being the case today—of course, with the added complications of digital and worldwide-distributed technologies. Research into correct implementations and deployments of these tools, possible attacks, improvements, enhancements in user experience, adaptation to modern more technologies and scenarios, and other relevant questions, is of high importance.


\part{安全多方计算基础} \label{part:1}

\chapter{安全多方计算的理论}\label{chp:01theory}

\section{安全多方计算概述} \label{1.1}


\chapter{基于秘密共享的安全多方计算}\label{chp:02ssmpc}

% \section{Reed-Solomon指纹识别}\label{2.1}
% \section{Freivalds算法}\label{2.2}
% \section{看待指纹识别和Freivalds算法的另一种视角}\label{2.3}
% \section{单变量Lagrange插值法}\label{2.4}

\part{诚实多数}
\chapter{Shamir秘密共享}\label{chp:03Shamir}
\section{秘密共享和$d$-一致性}\label{3.1}


% \section{论证系统}\label{3.2}
% \section{定义的健壮性及交互式的威力} \label{3.3}
% \section{Schwartz-Zippel引理}\label{3.4}
% \section{低次和多线性扩展}\label{3.5}
% \section{习题}\label{3.6}

\chapter{诚实多数被动完美安全}\label{chp:04pphsecure}
% \section{和校验协议}\label{4.1}
% \section{和校验的首个应用:$^\# \mathrm{SAT} \in \mathbf{I P}$}\label{4.2}
% \section{第二个应用:一个清点图中三角形数量的简单IP}\label{4.3}
% \section{第三个应用:对矩阵乘法的非常高效的IP}\label{4.4}
% \section{对矩阵乘法的非常高效的IP的应用}\label{4.5}
% \section{GKR协议及其高效实现}\label{4.6}
% \section{习题}\label{4.7}
\chapter{三分之二诚实多数主动完美安全}\label{chp:05ap23hsecure}

% \section{随机预言模型} \label{5.1}
% \section{Fiat-Shamir 转换} \label{5.2}
% \section{转换的安全性} \label{5.3}
% \section{习题} \label{5.4}

\chapter{诚实多数主动统计安全}\label{chp:06ashsecure}

% \section{引言}\label{6.1}
% \section{机器代码}\label{6.2}
% \section{将程序转换为电路的第一个技术(概览)}\label{6.3}
% \section{将小空间程序转换为浅电路}\label{6.4}
% \section{将计算机程序转换为电路可满足性实例}\label{6.5}
% \section{另外的转换和优化}\label{6.6}
% \section{习题}\label{6.7}

\part{不诚实多数}
\chapter{不诚实多数被动安全}\label{chp:07pdsecure}

% \section{朴素方法:一个对电路可满足性的IP}\label{7.1}
% \section{对电路可满足性的简洁论证}\label{7.2}
% \section{第一个对电路可满足性的简洁论证}\label{7.3}
% \section{知识可靠性}\label{7.4}
\chapter{不诚实多数主动安全}\label{chp:08adsecure}
% \section{一个对电路可满足性的高效MIP}\label{8.1}
% \section{一个对电路可满足性的高效MIP}\label{8.2}
% \section{一个对深电路的简洁论证}\label{8.3}
% \section{从Circuit-SAT到R1CS-SAT的扩展}\label{8.4}
% \section{MIP = NEXP}\label{8.5}




\backmatter

%%%%%%%%%%%%%%% Reference %%%%%%%%%%%%%%%

\printbibliography[heading=bibintoc]
\printindex

\end{document}

